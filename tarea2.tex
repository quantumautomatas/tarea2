%Margenes, idioma y tipo de documento
\documentclass{article}
\usepackage[spanish]{babel}
\usepackage[utf8]{inputenc}
\usepackage[margin = 1.5cm]{geometry}


\begin{document}
    
    %Titulo
    \title{Autómatas y Lenguajes formales 2019-2\\
    \large Ejercicio Semanal 2}

    \date{Fecha de entrega: 7 de febrero del 2019}

    \author{Sandra del Mar Soto Corderi\\
    Edgar Quiroz Castañeda}

    \maketitle

    %Problemas

    \begin{enumerate}
        \item {
            Una cadena es palíndorma si es de la forma $ww^R$.
            \begin{enumerate}
                \item {
                    Define el conjunto de cadenas palíndromas de forma recursiva.
                }
                \item {
                    Enuncia el pricnipio de Inducción generado por la definición
                    del inciso anterior.
                }
                \item {
                    Demuestra mediante inducción estrucutral que todas las 
                    cadenas palíndromas \textbf{de ese tipo} tienen un número 
                    par de símbolos
                }
                
            \end{enumerate}
        }
        \item {
            Dada la siguiente definición de árbol binario de elementos del 
            conjunto $A$
            \begin{itemize}
                \item {
                    Un árbol vacío $void$ es un árbol binario.
                }
                \item {
                    Si $T_1$ y $T_2$ son árboles binarios, y $c$ es un elemento 
                    de $A$, entonces $(tree \ T_1, c, T_2)$ es un árbol binario.
                }
                \item {
                    Estos son todos los árboles binarios.
                }
            \end{itemize}
            Responde a lo siguiente
            \begin{enumerate}
                \item {
                    Define recursivamente la función $aplana$ que toma
                    un árbol binario y devuelve una lista de sus elementos con 
                    un recorrido $\textit{in order}$.\\
		            
		            Usando la definición de lista de objetos que está en las notas, podemos definir $aplana$ como:\\
                    
                    $aplana$ : Árbol binario c $\rightarrow \mathcal{L}$(c)\\
					$aplana$ ($void$) = $nil$\\
					$aplana$ $(tree \ T_1, c, T_2)$ = $cons(cons(aplana(T_1), c), aplana(T_2))$\\
					
En haskell sería algo así:
					
aplana :: Arbol a $\rightarrow $ [a]\\
aplana void = []\\
aplana (tree a t1 t2) = aplana t1 ++ [a] ++ aplana t2	\\				
					
                }
                \item {
                    Demuestre que para cualquier árbol binario $T$ se cumple que
                    el número de nodos de $T$ es igual a la longitud de la lista 
                    $aplana(T)$.
                    
                    Demostramos realizando inducción sobre $T$ y usando la definición de lista de objetos que está en las notas:\\
                    Sea $n$ el número de nodos de $T$
                    \begin{itemize}                    
                    \item {
                        Caso base: $T = void$\\
                        $n$ = 0 por la definición de árbol binario\\
                        $aplana(T) =  nil$ por la definición de $aplana$ \\
                        $len(nil) = 0$ por la definición de longitud en listas\\ Entonces como los nodos y la longitud de la lista son 0, se cumple el caso base.
                        
                    }
					\item {
                       Hipótesis de inducción:
                       Suponemos verdadero que $n$ = $len(aplana(T))$ para los árboles $T_1$ y $T_2$\\
                                           }
                    Sea $n_1$ el número de nodos de $T_1$ y $n_2$ el número de nodos de $T_2$:
                    \item {
                        Paso inductivo, tenemos que demostrar que $len(aplana(tree \ T_1, c, T_2)) = n_1 + n_2 + 1$ \\ 
                        $aplana$ $(tree \ T_1, c, T_2)$ = $cons(cons(aplana(T_1), c), aplana(T_2))$ por definición de $aplana$\\
                        $len(cons(cons(aplana(T_1), c), aplana(T_2))) = len(cons(aplana(T_1), c)) + len(aplana(T_2)) $ por definición de longitud\\
                        $len(cons(aplana(T_1), c))$ = $len(aplana(T_1)) + 1$ por definición de longitud\\
                        De ahí, $len(aplana(tree \ T_1, c, T_2)) = len(aplana(T_1)) + len(aplana(T_2)) + 1 $ \\
                        Usando la hipótesis de inducción, $len(aplana(tree \ T_1, c, T_2)) = n_1 + n_2 + 1$ 
               }
                \end{itemize}
                Por lo tanto, se cumple  que el número de nodos de $T$ es igual a la longitud de la lista $aplana(T)$.
            }\\
                
            \end{enumerate}
        }
        \item {
            Define recursivamente la función $sub?(u, v)$ que nos dice si $u$ 
            es subcadena de $v$.
        }
        \item {
            Considera los lenguajes $L = \{ 0, 1, 10, 11, 100, 101, 111 \}$ y 
            $M = \{ 00, 01, 001, 11, 0100, 101, 0111 \}$.\\
            Da los lenguajes resultantes de cada inciso
            \begin{enumerate}
                \item {
                    $L \cup M = \{0, 1, 10, 11, 100, 101, 111, 00, 01, 001, 0100, 0111 \}$ 
                }
                \item {
                    $M \cap L^R = \{01, 001, 11, 101 \}$
                }
                \item {
                    $L - (M \cup L^R) = \{10, 100 \}$
                }
                \item {
                    $(ML)^R = \{000, 100, 0100, 1100, 00100, 10100, 11100, 010, 110, 0110, 1110, 00110, 10110, 11110, 01100, 001100, 101100,$\\
                     $111100, 011, 111, 0111, 1111, 00111, 10111, 11111, 00010, 10010, 010010, 110010, 0010010, 1010010, 1110010, 0101, 1101,$\\
                      $01101, 11101, 001101, 101101, 111101, 01110, 011110, 111110, 0011110, 1011110, 1111110 \}$
                }
                \item {
                    $L^RM^R = \{000, 100, 0100, 1100, 00100, 10100, 11100, 010, 110, 0110, 1110, 00110, 10110, 11110, 01100, 001100, 101100,$\\
                     $111100, 011, 111, 0111, 1111, 00111, 10111, 11111, 00010, 10010, 010010, 110010, 0010010, 1010010, 1110010, 0101, 1101,$\\
                      $01101, 11101, 001101, 101101, 111101, 01110, 011110, 111110, 0011110, 1011110, 1111110 \}$
                                                                  
   }
            \end{enumerate}
        }
    \end{enumerate}

\end{document}