%Tipo de formato
\documentclass{article}

%Ajustar márgenes
\usepackage[margin = 1cm]{geometry}

%Algunas símbolos matemáticos (letras caligráficas)
\usepackage{amssymb}


\begin{document}
    
    %Titulo
    \title{Autómatas y Lenguajes formales 2019-2\\
    \large Ejercicio Semanal 2}

    \date{Fecha de entrega: 7 de febrero del 2019}

    \author{Sandra del Mar Soto Corderi\\
    Edgar Quiroz Castañeda}

    \maketitle

    %Problemas

    \begin{enumerate}
        \item {
            Una cadena es palíndroma si es de la fomra $ww^R$.
            \begin{enumerate}
                \item {
                    Define el conjunto $P$ de cadenas palíndromas de forma 
                    recursiva.
                    \begin{itemize}
                        \item {
                            $\epsilon$ es cadena palíndroma.
                        }
                        \item {
                            Si $a \in \Sigma$ y $p \in P$, entonces $apa \in P$
                        }
                        \item {
                            Estos son todos las cadenas palíndromas 
                        }
                    \end{itemize}
                }
                \item {
                    Enuncia el principio de Inducción generado por la definición
                    del inciso anterior.\\
                    Sea $Pr$ alguna propiedad. Para demostrar que $Pr$ se cumple
                    para todas las cadenas paíndromas, hay que hacer lo siguiente
                    \begin{itemize}
                        \item {
                            Demostrar directamente que $Pr(\epsilon)$ es verdad.
                        }
                        \item {
                            Asumir como hipótesis que si alguna cadena 
                            palíndroma es de la forma, $apa$, entonces $Pr(p)$ es verdad.
                        }
                        \item {
                            Demostrar que $P(apa)$ es verdad usando la 
                            hipótesis del inciso anterior.
                        }
                    \end{itemize}
                }
                \item {
                    Demuestra mediante inducción estrucutral que todas las 
                    cadenas palíndromas \textbf{de ese tipo} tienen un número 
                    par de símbolos.
                    \begin{itemize}
                        \item {
                            Caso base, con $w = \epsilon$.\\
                            Entonces $|w| = 0 = 2\cdot 0$.\\
                            Por lo que $w = \epsilon$ tiene una cantidad par de 
                            símbolos.
                        }
                        \item {
                            La hipótesis es que si $awa$ es cadena palíndorma, 
                            entonces $w$ tiene una cantidad par de símbolos, 
                            esto es que $|w| = 2k$ para alguna $k \in \mathbb{N}$
                        }
                        \item {
                            Notemos que $a \in \Sigma$ es un símbolo, para 
                            también puede considerarse una cadena de un sólo 
                            símbolo, por lo que las propiedades válidas para 
                            cadenas también son válidas para $a$.\\
                            En particular, tenemos que $|vw| = |v| + |w|$ 
                            (demostrado en la tarea anterior).\\
                            Por lo que
                            \[|awa| = |a| + |aw| = |a| + |w| + |a| = 1 + |w| + 1
                            = |w| + 2|\]
                            Luego, por la hipótesis de inducción, $|w| = 2k$.\\
                            Por lo que $|awa| = 2k + 2 = 2 (k+1)$, que es un 
                            número par.\\
                            Por lo que $awa$ tiene una cantidad par de símbolos.                        }
                    \end{itemize}
                }
                
            \end{enumerate}
        }
        \item {
            Dada la siguiente definición de árbol binario de elementos del 
            conjunto $A$
            \begin{itemize}
                \item {
                    Un árbol vacío $void$ es un árbol binario.
                }
                \item {
                    Si $T_1$ y $T_2$ son árboles binarios, y $c$ es un elemento 
                    de $A$, entonces $(T_1, c, T_2)$ es un árbol binario.
                }
                \item {
                    Estos son todos los árboles binarios.
                }
            \end{itemize}
            Responde a lo siguiente
            \begin{enumerate}
                \item {
                    Define recursivamente la función $aplana$ que toma
                    un árbol binario y devuelve una lista de sus elementos con 
                    un recorrido $\textit{in order}$.
                }
                \item {
                    Demuestre que para cualquer árbol binario $T$ se cumple que
                    el número de nodos de $T$ es igual a la longitud de la lista 
                    $aplana(T)$.
                }
            \end{enumerate}
        }
        \item {
            Define recursivamente la función $sub?(u, v)$ que nos dice si $u$ 
            es subcadena de $v$.\\
            Tenemos que $sub?: \Sigma^{*2} \rightarrow \{0, 1\}$ dada por
            \begin{itemize}
                \item {
                    $sub? (\epsilon, v) = 1$
                }
                \item {
                    $sub? (u, \epsilon) = 0$
                }
                \item {
                    $sub?(\alpha u, \beta v) = ((\alpha == \beta) 
                    \wedge sub?(u, v)) \vee (sub?(\alpha u, v))$
                }
            \end{itemize}
        }
        \item {
            Considera los lenguajes $L = \{ 0, 1, 10, 11, 100, 101, 111 \}$ y 
            $M = \{ 00, 01, 001, 11, 0100, 101, 0111 \}$.\\
            Da los lenguajes resultantes de cada inciso
            \begin{enumerate}
                \item {
                    $L \cup M$
                }
                \item {
                    $M \cap L^R$
                }
                \item {
                    $L - (M \cup L^R)$
                }
                \item {
                    $(ML)^R$
                }
                \item {
                    $L^2M^2$
                }
            \end{enumerate}
        }
    \end{enumerate}

\end{document}