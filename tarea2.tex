%Margenes, idioma y tipo de documento
\documentclass{article}
\usepackage[spanish]{babel}
\usepackage[utf8]{inputenc}
\usepackage[margin = 1.5cm]{geometry}

%Algunas símbolos matemáticos (letras caligráficas)
\usepackage{amssymb}
\usepackage{amsmath}


\begin{document}
    
    %Titulo
    \title{Autómatas y Lenguajes formales 2019-2\\
    \large Ejercicio Semanal 2}

    \date{Fecha de entrega: 7 de febrero del 2019}

    \author{Sandra del Mar Soto Corderi\\
    Edgar Quiroz Castañeda}

    \maketitle

    %Problemas

    \begin{enumerate}
        \item {

            Una cadena es palíndorma si es de la forma $ww^R$.
            
            \begin{enumerate}
                \item {
                    Define el conjunto $P$ de cadenas palíndromas de forma 
                    recursiva.
                    \begin{itemize}
                        \item {
                            $\epsilon$ es cadena palíndroma.
                        }
                        \item {
                            Si $a \in \Sigma$ y $p \in P$, entonces $apa \in P$
                        }
                        \item {
                            Estos son todos las cadenas palíndromas 
                        }
                    \end{itemize}
                }
                \item {
                    Enuncia el principio de Inducción generado por la definición
                    del inciso anterior.\\
                    Sea $Pr$ alguna propiedad. Para demostrar que $Pr$ se cumple
                    para todas las cadenas paíndromas, hay que hacer lo siguiente
                    \begin{itemize}
                        \item {
                            Demostrar directamente que $Pr(\epsilon)$ es verdad.
                        }
                        \item {
                            Asumir como hipótesis que si alguna cadena 
                            palíndroma es de la forma, $apa$, entonces $Pr(p)$ es verdad.
                        }
                        \item {
                            Demostrar que $P(apa)$ es verdad usando la 
                            hipótesis del inciso anterior.
                        }
                    \end{itemize}
                }
                \item {
                    Demuestra mediante inducción estrucutral que todas las 
                    cadenas palíndromas \textbf{de ese tipo} tienen un número 
                    par de símbolos.
                    \begin{itemize}
                        \item {
                            Caso base, con $w = \epsilon$.\\
                            Entonces $|w| = 0 = 2\cdot 0$.\\
                            Por lo que $w = \epsilon$ tiene una cantidad par de 
                            símbolos.
                        }
                        \item {
                            La hipótesis es que si $awa$ es cadena palíndorma, 
                            entonces $w$ tiene una cantidad par de símbolos, 
                            esto es que $|w| = 2k$ para alguna $k \in \mathbb{N}$
                        }
                        \item {
                            Notemos que $a \in \Sigma$ es un símbolo, para 
                            también puede considerarse una cadena de un sólo 
                            símbolo, por lo que las propiedades válidas para 
                            cadenas también son válidas para $a$.\\
                            En particular, tenemos que $|vw| = |v| + |w|$ 
                            (demostrado en la tarea anterior).\\
                            Por lo que
                            \[|awa| = |a| + |aw| = |a| + |w| + |a| = 1 + |w| + 1
                            = |w| + 2|\]
                            Luego, por la hipótesis de inducción, $|w| = 2k$.\\
                            Por lo que $|awa| = 2k + 2 = 2 (k+1)$, que es un 
                            número par.\\
                            Por lo que $awa$ tiene una cantidad par de símbolos.                        }
                    \end{itemize}
                }
                
            \end{enumerate}
        }
        \item {
            Dada la siguiente definición de árbol binario de elementos del 
            conjunto $A$
            \begin{itemize}
                \item {
                    Un árbol vacío $void$ es un árbol binario.
                }
                \item {
                    Si $T_1$ y $T_2$ son árboles binarios, y $c$ es un elemento 
                    de $A$, entonces $(tree \ T_1, c, T_2)$ es un árbol binario.
                }
                \item {
                    Estos son todos los árboles binarios.
                }
            \end{itemize}
            Responde a lo siguiente
            \begin{enumerate}
                \item {
                    Define recursivamente la función $aplana$ que toma
                    un árbol binario y devuelve una lista de sus elementos con 
                    un recorrido $\textit{in order}$.
		            
                    Usando la definición de lista de objetos que está en las notas, 
                    podemos definir $aplana$ como:
                    
                    $aplana$ :: BinTree a $\rightarrow$ [a]\\
					$aplana$ ($void$) = []\\
                    $aplana$ $(tree \ T_1, e, T_2)$ = 
                    $(aplana \ T_1) ++ (e:(aplana \ T_2))$

                    Donde $++$ es el operador de concatenación en listas.
                }
                \item {
                    Demuestre que para cualquier árbol binario $T$ se cumple que
                    el número de nodos de $T$ es igual a la longitud de la lista 
                    $aplana(T)$.
                    
                    Primero, consideremos la siguiente función que da la cantidad
                    de nodos de un árbol:\\ 
                    $|\cdot|$ :: BinTree a $\rightarrow$ $\mathbb{N}$\\
					$|void|$ = 0\\
					$|(tree \ T_1, e, T_2)| = |T_1| + |T_2| + 1$
                    
                    
                    Ahora, demostremos que $|T| = len(aplana(T))$ por inducción 
                    sobre $T$.
                    \begin{itemize}                    
                    \item {
                        Caso base: $T = void$\\
                        $|void|$ = 0 por la definición de $|\cdot|$\\
                        $aplana(T) = []$ por la definición de $aplana$ \\
                        $len(aplana(T)) = len([]) = 0 = |void| = |T|$ por la 
                        definición de longitud en listas.
                    }
					\item {
                        Hipótesis: \\
                        Sea $(tree \ T_1, e, T_2)$ árbol binario. Entonces, 
                        $|T_1| = len(aplana(T_1))$ y $|T_2| = len(aplana(T_2))$ 
                    }
                    \item {
                        Hay que demostrar que 
                        $|(tree \ T_1, e, T_2)| = len(aplana \ (tree \ T_1, e, T_2))$.\\
                        Tenemos que $aplana(tree \ T_1, e, T_2) = 
                        (aplana \ T_1) ++ (e:(aplana \ T_2))$, por definición de 
                        aplana.\\
                        Luego, $len((aplana \ T_1) ++ (e:(aplana \ T_2))) 
                        = len(aplana \ T_1) + len(e:(aplana \ T_2))$ por 
                        propiedad de la concatenación en listas.\\
                        Y por definición de longitud en listas, 
                        $len(e:(aplana \ T_2)) = 1 + len(aplana \ T_2)$.\\
                        Y usando la hipótesis, tenemos que 
                        $len(aplana \ T_1) = |T_1|$ y $len(aplana \ T_2) = |T_2|$.\\
                        Por lo que $len(aplana (tree \ T_1, e, T_2)) 
                        = |T_1| + 1 + |T_2| = |(tree \ T_1, e, T_2)|$,
                        por definición de $|\cdot|$.
                    }
                \end{itemize}
                Por lo tanto, se cumple  que el número de nodos de $T$ es 
                igual a la longitud de la lista $aplana(T)$.
            }
                
            \end{enumerate}
        }
        
        \item {
            Considera los lenguajes $L = \{ 0, 1, 10, 11, 100, 101, 111 \}$ y 
            $M = \{ 00, 01, 001, 11, 0100, 101, 0111 \}$.\\
            Da los lenguajes resultantes de cada inciso
            \begin{enumerate}
                \item {
                    $L \cup M = \{0, 1, 10, 11, 100, 101, 111, 00, 01, 001, 0100, 0111 \}$ 
                }
                \item {
                    $M \cap L^R = \{01, 001, 11, 101 \}$
                }
                \item {
                    $L - (M \cup L^R) = \{10, 100 \}$
                }
                \item {
                    $(ML)^R$ = \{000, 100, 0100, 1100, 00100, 10100, 11100, 010, 
                    110, 0110, 1110, 00110, 10110, 11110, 01100,001100, 101100, 
                    11100, 011, 111, 0111, 1111, 00111, 10111, 11111, 00010, 10010, 
                    010010, 110010, 0010010, 1010010, 1110010, 0101, 1101, 01101, 
                    11101, 001101, 101101, 111101, 01110, 011110, 111110, 0011110, 
                    1011110, 1111110\}
                }
                \item {
                    $L^2M^R$ = \{0000, 0010, 00100, 0011, 000010, 00101, 001110, 
                    0100, 0110, 01100, 0111, 010010, 01101, 011110, 01000, 01010,
                    010100, 01011, 0100010, 010101, 0101110, 01110, 011100, 01111, 
                    0110010, 011101, 0111110, 010000, 0100100, 010011, 01000010, 
                    0100101, 01001110, 010110, 0101100, 010111, 01010010, 0101101, 
                    01011110, 0111100, 011111, 01110010, 0111101, 01111110, 1000, 
                    1010, 10100, 1011, 100010, 10101, 101110, 1100, 1110, 11100, 
                    1111, 110010, 11101, 111110, 11000, 11010, 110100, 11011, 
                    1100010, 110101, 1101110, 11110, 111100, 11111, 1110010, 111101,
                    1111110, 110000, 1100100, 110011, 11000010, 1100101, 11001110, 
                    110110, 1101100, 110111, 11010010, 1101101, 11011110, 1111100, 
                    111111, 11110010, 1111101, 11111110, 10000, 10010, 100100, 10011, 
                    1000010, 100101, 1001110, 10110, 101100, 10111, 1010010, 101101,
                    1011110, 101000, 101010, 1010100, 101011, 10100010, 1010101, 
                    10101110, 1011100, 101111, 10110010, 1011101, 10111110, 1010000,
                    10100100, 1010011, 101000010, 10100101, 101001110, 1010110, 
                    10101100, 1010111, 101010010, 10101101, 101011110, 10111100, 
                    1011111, 101110010, 10111101, 101111110, 111000, 111010, 
                    1110100, 111011, 11100010, 1110101, 11101110,1110000, 11100100,
                    1110011, 111000010, 11100101, 111001110, 1110110, 11101100, 
                    1110111, 111010010, 11101101, 111011110, 11111100, 1111111, 
                    111110010, 11111101, 111111110,100000, 1000100, 100011, 10000010, 
                    1000101, 10001110, 100110, 1001100, 100111, 10010010, 1001101, 
                    10011110, 1001000, 1001010, 10010100, 1001011, 100100010, 
                    10010101, 100101110, 10011100, 1001111, 100110010, 10011101, 
                    100111110, 10010000, 100100100,10010011, 1001000010, 100100101, 
                    1001001110, 10010110, 100101100, 10010111, 1001010010, 100101101, 
                    1001011110, 100111100, 10011111, 1001110010, 100111101, 
                    1001111110,1011000, 1011010, 10110100, 1011011, 101100010, 
                    10110101, 101101110, 10110000,101100100, 10110011, 1011000010, 
                    101100101, 1011001110, 10110110, 101101100, 10110111, 1011010010, 
                    101101101, 1011011110, 101111100, 10111111, 1011110010, 101111101,
                    1011111110, 1111000, 1111010, 11110100, 1111011, 111100010, 
                    11110101, 111101110,11110000, 111100100, 11110011, 1111000010,
                    111100101, 1111001110, 11110110, 111101100,11110111, 1111010010, 
                    111101101, 1111011110, 111111100, 11111111, 1111110010,111111101, 
                    1111111110\}
                }
            \end{enumerate}
        }
    \end{enumerate}

\end{document}