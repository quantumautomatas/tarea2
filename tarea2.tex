%Tipo de formato
\documentclass{article}

%Ajustar márgenes
\usepackage[margin = 1cm]{geometry}


\begin{document}
    
    %Titulo
    \title{Autómatas y Lenguajes formales 2019-2\\
    \large Ejercicio Semanal 2}

    \date{Fecha de entrega: 7 de febrero del 2019}

    \author{Sandra del Mar Soto Corderi\\
    Edgar Quiroz Castañeda}

    \maketitle

    %Problemas

    \begin{enumerate}
        \item {
            Una cadena es palíndorma si es de la fomra $ww^R$.
            \begin{enumerate}
                \item {
                    Define el conjunto de cadenas palíndromas de forma recursiva.
                }
                \item {
                    Enuncia el pricnipio de Inducción generado por la definición
                    del inciso anterior.
                }
                \item {
                    Demuestra mediante inducción estrucutral que todas las 
                    cadenas palíndromas \textbf{de ese tipo} tienen un número 
                    par de símbolos
                }
                
            \end{enumerate}
        }
        \item {
            Dada la siguiente definición de árbol binario de elementos del 
            conjunto $A$
            \begin{itemize}
                \item {
                    Un árbol vacío $void$ es un árbol binario.
                }
                \item {
                    Si $T_1$ y $T_2$ son árboles binarios, y $c$ es un elemento 
                    de $A$, entonces $(T_1, c, T_2)$ es un árbol binario.
                }
                \item {
                    Estos son todos los árboles binarios.
                }
            \end{itemize}
            Responde a lo siguiente
            \begin{enumerate}
                \item {
                    Define recursivamente la función $aplana$ que toma
                    un árbol binario y devuelve una lista de sus elementos con 
                    un recorrido $\textit{in order}$.
                }
                \item {
                    Demuestre que para cualquer árbol binario $T$ se cumple que
                    el número de nodos de $T$ es igual a la longitud de la lista 
                    $aplana(T)$.
                }
            \end{enumerate}
        }
        \item {
            Define recursivamente la función $sub?(u, v)$ que nos dice si $u$ 
            es subcadena de $v$.
        }
        \item {
            Considera los lenguajes $L = \{ 0, 1, 10, 11, 100, 101, 111 \}$ y 
            $M = \{ 00, 01, 001, 11, 0100, 101, 0111 \}$.\\
            Da los lenguajes resultantes de cada inciso
            \begin{enumerate}
                \item {
                    $L \cup M$
                }
                \item {
                    $M \cap L^R$
                }
                \item {
                    $L - (M \cup L^R)$
                }
                \item {
                    $(ML)^R$
                }
                \item {
                    $L^2M^2$
                }
            \end{enumerate}
        }
    \end{enumerate}

\end{document}